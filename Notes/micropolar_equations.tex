%\documentclass[12pt,letterpaper]{article}
%\usepackage[utf8]{inputenc}
%\usepackage{amsmath,
%			amsfonts,
%			amssymb}
%\usepackage{graphicx}
%\usepackage{subcaption}
%\usepackage{float}
%\usepackage{physics}
%\usepackage{cancel}
%
%
%
%
%
%\author{Gary Dargush, Ali Reza Hadjesandiari\\Nicolás Guarín-Zapata and Juan Gomez}
%\title{\textbf{Revision o the continuum mechanics consistency of micropolar field theories from the wave propagation point of view}}


%%%%%
\documentclass[12pt]{article}
\usepackage{geometry}
\usepackage{amsmath, amsfonts, amssymb}
\usepackage[utf8]{inputenc}
\usepackage{graphicx}
\usepackage{geometry}
\usepackage[round]{natbib}
\usepackage{subfig} % Subgraphics
\usepackage{hyperref} % Nice PDF
\usepackage{physics}
\usepackage{float}

\geometry{
	verbose,
	letterpaper}

\hypersetup{
	bookmarksopen=true,
	colorlinks=true,
	linkcolor=blue,
	citecolor=blue,
	urlcolor=black,	
	linktoc=all,
	pdftitle={Waves in Cosserat media},
	pdfauthor={N. Guarin-Zapata, J. Gomez},
	pdfkeywords={Finite element method, Cosserat solids, Bloch theorem, Wave propagation},
	pdfsubject={Elastodynamics},
	pdfpagemode=UseOutlines,
	pdfstartview=FitH}


\title{\textbf{The micropolar continuum theory: a revision from the wave propagation perspective}}

\author{
	Gary Dargush, Ali Reza Hadjesfandiari\\Nicolás Guarín-Zapata and Juan Gomez\\}




%%%%%%




\begin{document}
\maketitle

\section{Introduction}
The classical model of continuum mechanics lacks a length scale thus it is unable to capture so-called size effects. In the early 1900s, starting with the work from the Cosserat brothers \citep{Cosserat1909} a series of non-classical models were proposed with an intrinsic capability to account for microstructural effects while retaining a continuum approach. Broadly speaking these models could be categorized into micropolar models incorporating additional degrees of freedom and reduced or gradient models keeping additional displacement gradients in its kinematic formulation. Although most of these formulations are mathematically consistent and are also able to predict numerical results valid under particular scenarios, not all of them are fully consistent as a continuum theory. For instance in a series of recent contribution Hadjesfandiari and Dargush() and Dargush and Hadjesfandiari() presented arguments questioning the continuum mechanics consistency of non-classical models. Moreover, these authors (see Hadjesfandiari and Dargush() ) formulated a valid Couple stress framework after fixing the kinematical indeterminacy in the reduced couple stress theory.

Among the non-classical models, the micropolar theory \citep{Eringen1966} is very convenient since it just introduces an additional rotation independent of the macro-rotation which conserves the original structure of the displacements equations from classical theory and also eliminates compatibility issues in the corresponding numerical implementations. In this work we conduct additional scrutiny o the micropolar model but examining it from the point of view of its response under dynamic conditions and particularly from its wave propagation perspective. We start by reviewing the stress and displacement equations of motion and in general the mathematical consistency of the problem. Using vector potentials we find the existent propagation modes, derive dispersion relations and reflection coefficients. As a final verification we use Bloch analysis to obtain the band structure of a homogeneous micropolar model.



\section{Fundamental relationships}
\subsection{Conservation of linear and angular momentum}
In micropolar elasticity we have conservation of linear and angular momentum given by the following expressions
\begin{subequations}\label{eq:conservation}
  \begin{align}
    &\sigma_{ji, i} + f_i = \rho \ddot{u}_i\\
    &\sigma_{jk} \epsilon_{ijk} + \mu_{ji, j} + c_i = J \ddot{\varphi}\, .
  \end{align}
\end{subequations}

\subsection{Kinematic relations}
The (linearized) kinematic relations are given by
\begin{subequations}\label{eq:kinematics}
  \begin{align}
    & \gamma_{ji} = u_{i,j} - \epsilon_{kji} \varphi_k\\
    & \kappa_{ji} = \varphi_{i,j}\, .
  \end{align}
\end{subequations}

\subsection{Constitutive equations}
In the linear regime, the constitutive equations are
\begin{subequations}\label{eq:constitutive}
  \begin{align}
    & \sigma_{ji} = (\mu + \alpha) \gamma_{ji} + (\mu - \alpha) \gamma_{ij} + \lambda \gamma_{kk} \delta_{ij}\\
    & \mu_{ji} = (\gamma + \epsilon) \kappa_{ji} + (\gamma - \epsilon) \kappa_{ij} + \beta \kappa_{kk} \delta_{ij}\, .
  \end{align}
\end{subequations}

\section{Derivation of the displacement equations}
We can start writing the conservation equations \eqref{eq:conservation} as kinematic variables using the constitutive equations \eqref{eq:constitutive}, to obtain:
\begin{align*}
  & (\mu + \alpha) \gamma_{ji, j} + (\mu - \alpha) \gamma_{ij,j} + \lambda \gamma_{kk, i} + f_i = \rho \ddot{u}_i\\
  & \epsilon_{ijk}(\mu + \alpha) \gamma_{jk} + \epsilon_{ijk}(\mu - \alpha) \gamma_{kj} + \cancelto{0}{\epsilon_{ijk} \delta_{jk} \lambda \gamma_{rr}} + (\gamma + \epsilon) \kappa_{ji,j} + (\gamma - \epsilon) \kappa_{ij,j} + \beta \kappa_{kk, i} + c_i = J \ddot{\varphi}_i\, .
\end{align*}

Let's focus in the equations for the conservation of linear momentum first. If we replace the kinematics relations, we get
\[(\mu + \alpha) u_{i, jj} - (\mu + \alpha)\epsilon_{kji} \varphi_{k, j} + (\mu - \alpha) u_{j, ij} + (\mu - \alpha)\epsilon_{kji} \varphi_{k,j} + \lambda u_{k,ki} - \cancelto{0}{\epsilon_{krr}\lambda \varphi_{k,j}} + f_i = \rho \ddot{u}_i\, , \]
or
\[(\mu + \alpha) u_{i,jj} + (\mu - \alpha) u_{j,ij} + \lambda u_{k,kj} - 2\alpha \epsilon_{kji} + f_i = \rho \ddot{u}_i\, .\]

And, using the identity \(a_{i,jj} = a_{j,ji} - \epsilon_{ijk} \epsilon_{klm} a_{m,lj}\),
\[(\mu + \alpha) u_{k, ki} - \epsilon_{ijk} \epsilon_{klm} (\mu + \alpha) u_{m, lj} + (\mu - \alpha) u_{j,ij} + \lambda u_{k, ki} - 2\alpha \epsilon_{kji} \varphi_{k,j} + f_i = \rho \ddot{u}_i\, ,\]
grouping \(u_{k,ki}\),
\[(\mu + \alpha + \lambda) u_{k, ki} - \epsilon_{ijk} \epsilon_{klm} (\mu + \alpha) u_{m,lj} + (\mu - \alpha) u_{j, ij} + 2\alpha \epsilon_{ijk} \varphi_{k,j} + f_i = \rho \ddot{u}_i\, ,\]
but \(u_{j, ij} = u_{j, ji}\), thus
\begin{equation}
  (\lambda + 2\mu) u_{k, ki} - \epsilon_{ijk} \epsilon_{klm} (\mu + \alpha) u_{m,lj}+ 2\alpha \epsilon_{ijk} \varphi_{k,j} + f_i = \rho \ddot{u}_i\, .
\end{equation}

\bibliographystyle{unsrtnat}
\bibliography{bloch-cosserat}

\end{document}