\documentclass[12pt]{article}
\usepackage{geometry}
\usepackage{amsmath, amsfonts, amssymb}
\usepackage[utf8]{inputenc}
\usepackage{graphicx}
\usepackage[round]{natbib}
\usepackage{subfig}
\usepackage{hyperref}
\usepackage{physics}
\usepackage{float}
\usepackage{cancel}


\hypersetup{
	bookmarksopen=true,
	colorlinks=true,
	linkcolor=blue,
	citecolor=blue,
	urlcolor=black,	
	linktoc=all,
	pdftitle={Waves in Cosserat media},
	pdfauthor={N. Guarin-Zapata, J. Gomez},
	pdfkeywords={Finite element method, Cosserat solids, Bloch theorem, Wave propagation},
	pdfsubject={Elastodynamics},
	pdfpagemode=UseOutlines,
	pdfstartview=FitH}


\title{\textbf{The micropolar continuum theory: a revision from the wave propagation perspective}}

\author{Gary Dargush \and Ali Reza Hadjesfandiari \and Nicolás Guarín-Zapata \and Juan Gomez}


\begin{document}
\maketitle

\section{Introduction}
The classical model of continuum mechanics lacks a length scale thus it is unable to capture so-called size effects. In the early 1900s, starting with the work from the Cosserat brothers \citep{Cosserat1909} a series of non-classical models were proposed with an intrinsic capability to account for microstructural effects while retaining a continuum approach. Broadly speaking these models could be categorized into micropolar models incorporating additional degrees of freedom and reduced or gradient models keeping additional displacement gradients in its kinematic formulation. Although most of these formulations are mathematically consistent and are also able to predict numerical results valid under particular scenarios, not all of them are fully consistent as a continuum theory. For instance in a series of recent contribution Hadjesfandiari and Dargush() and Dargush and Hadjesfandiari() presented arguments questioning the continuum mechanics consistency of non-classical models. Moreover, these authors (see Hadjesfandiari and Dargush() ) formulated a valid Couple stress framework after fixing the kinematical indeterminacy in the reduced couple stress theory.

Among the non-classical models, the micropolar theory \citep{Eringen1966} is very convenient since it just introduces an additional rotation independent of the macro-rotation which conserves the original structure of the displacements equations from classical theory and also eliminates compatibility issues in the corresponding numerical implementations. In this work we conduct additional scrutiny o the micropolar model but examining it from the point of view of its response under dynamic conditions and particularly from its wave propagation perspective. We start by reviewing the stress and displacement equations of motion and in general the mathematical consistency of the problem. Using vector potentials we find the existent propagation modes, derive dispersion relations and reflection coefficients. As a final verification we use Bloch analysis to obtain the band structure of a homogeneous micropolar model.



\section{Fundamental relationships}
\subsection{Conservation of linear and angular momentum}
In micropolar elasticity we have conservation of linear and angular momentum given by the following expressions
\begin{subequations}\label{eq:conservation}
  \begin{align}
    &\sigma_{ji, i} + f_i = \rho \ddot{u}_i\\
    &\sigma_{jk} \epsilon_{ijk} + \mu_{ji, j} + c_i = J \ddot{\varphi}\, .
  \end{align}
\end{subequations}

\subsection{Kinematic relations}
The (linearized) kinematic relations are given by
\begin{subequations}\label{eq:kinematics}
  \begin{align}
    & \gamma_{ji} = u_{i,j} - \epsilon_{kji} \varphi_k\\
    & \kappa_{ji} = \varphi_{i,j}\, .
  \end{align}
\end{subequations}

\subsection{Constitutive equations}
In the linear regime, the constitutive equations are
\begin{subequations}\label{eq:constitutive}
  \begin{align}
    & \sigma_{ji} = (\mu + \alpha) \gamma_{ji} + (\mu - \alpha) \gamma_{ij} + \lambda \gamma_{kk} \delta_{ij}\\
    & \mu_{ji} = (\gamma + \varepsilon) \kappa_{ji} + (\gamma - \varepsilon) \kappa_{ij} + \beta \kappa_{kk} \delta_{ij}\, .
  \end{align}
\end{subequations}

These constitutive equations can also be written as
\begin{align*}
    & \sigma_{ji} = \mu \gamma_{(ij)} + 2 \alpha \gamma_{[ij]} + \lambda \gamma_{kk} \delta_{ij}\\
    & \mu_{ji} = \gamma \kappa_{(ij)} + 2\varepsilon \kappa_{[ij]} + \beta \kappa_{kk} \delta_{ij}\, ,
\end{align*}
where we have separated the symmetric and antisymmetric parts of the 

\section{Derivation of the equations for displacements and rotations}
We can start writing the conservation equations \eqref{eq:conservation} as kinematic variables using the constitutive equations \eqref{eq:constitutive}, to obtain:
\begin{align*}
 & (\mu + \alpha) \gamma_{ji, j} + (\mu - \alpha) \gamma_{ij,j} + \lambda \gamma_{kk, i} + f_i = \rho \ddot{u}_i\\
  & \epsilon_{ijk}(\mu + \alpha) \gamma_{jk} + \epsilon_{ijk}(\mu - \alpha) \gamma_{kj} + \cancelto{0}{\epsilon_{ijk} \delta_{jk} \lambda \gamma_{rr}} + (\gamma + \epsilon) \kappa_{ji,j} + (\gamma - \epsilon) \kappa_{ij,j} + \beta \kappa_{kk, i} + c_i = J \ddot{\varphi}_i\, .
\end{align*}

Let's focus in the equations for the conservation of linear momentum first. If we replace the kinematics relations, we get
\[(\mu + \alpha) u_{i, jj} - (\mu + \alpha)\epsilon_{kji} \varphi_{k, j} + (\mu - \alpha) u_{j, ij} + (\mu - \alpha)\epsilon_{kji} \varphi_{k,j} + \lambda u_{k,ki} - \cancelto{0}{\epsilon_{krr}\lambda \varphi_{k,j}} + f_i = \rho \ddot{u}_i\, , \]
or
\[(\mu + \alpha) u_{i,jj} + (\mu - \alpha) u_{j,ij} + \lambda u_{k,kj} - 2\alpha \epsilon_{kji} + f_i = \rho \ddot{u}_i\, .\]

And, using the identity \(a_{i,jj} = a_{j,ji} - \epsilon_{ijk} \epsilon_{klm} a_{m,lj}\),
\[(\mu + \alpha) u_{k, ki} - \epsilon_{ijk} \epsilon_{klm} (\mu + \alpha) u_{m, lj} + (\mu - \alpha) u_{j,ij} + \lambda u_{k, ki} - 2\alpha \epsilon_{kji} \varphi_{k,j} + f_i = \rho \ddot{u}_i\, ,\]
grouping \(u_{k,ki}\),
\[(\mu + \alpha + \lambda) u_{k, ki} - \epsilon_{ijk} \epsilon_{klm} (\mu + \alpha) u_{m,lj} + (\mu - \alpha) u_{j, ij} + 2\alpha \epsilon_{ijk} \varphi_{k,j} + f_i = \rho \ddot{u}_i\, ,\]
but \(u_{j, ij} = u_{j, ji}\), thus
\[(\lambda + 2\mu) u_{k, ki} - \epsilon_{ijk} \epsilon_{klm} (\mu + \alpha) u_{m,lj}+ 2\alpha \epsilon_{ijk} \varphi_{k,j} + f_i = \rho \ddot{u}_i\, .\]

Following a similar approach, we obtain
\[(\beta + 2\gamma) \varphi_{k, ki} - \epsilon_{ijk} \epsilon_{klm} (\gamma + \varepsilon) u_{m,lj}+ 2\alpha \epsilon_{ijk} u_{k,j} - 4\alpha\varphi + c_i = J \ddot{\varphi}_i\, ,\]
for rotations.

\section{Equations in index and vector notation}
Authors presents the equations in slightly different ways, in this section we present the equations in two different forms.

One form is
\begin{subequations}
  \begin{align}
    & (\mu + \alpha) u_{i, jj} + (\lambda + \mu - \alpha) u_{j,ji} + 2\alpha \epsilon_{ijk}\varphi_{k,j} + f_i = \rho \ddot{u}_i\, \\
    & (\gamma + \varepsilon) \varphi_{i, jj} + (\beta + \gamma - \varepsilon) \varphi_{j,ji} + 2\alpha \epsilon_{ijk}u_{k,j} - 4\alpha \varphi_i  + c_i = J \ddot{\varphi}_i\, ,
  \end{align}
\end{subequations}
or, in vector notation
\begin{align*}
    & (\mu + \alpha)\laplacian{\vb{u}} + (\lambda + \mu - \alpha) \grad\div\vb{u} + 2\alpha \curl\vb{\varphi} + \vb{f} = \rho \pdv[2]{\vb{u}}{t}\, \\
    & (\gamma + \varepsilon) \laplacian\vb{\varphi} + (\beta + \gamma - \varepsilon) \grad\div\vb{\varphi} + 2\alpha \curl\vb{u} - 4\alpha \vb{\varphi}  + \vb{c} = J \pdv[2]{\varphi}{t}\, .
\end{align*}

We find the second form more appropriated for wave propagation. This one, reads
\begin{subequations}
  \begin{align}
    & (\lambda + 2\mu) u_{k, ki} - \epsilon_{ijk} \epsilon_{klm} (\mu + \alpha) u_{m,lj}+ 2\alpha \epsilon_{ijk} \varphi_{k,j} + f_i = \rho \ddot{u}_i, \\
    & (\beta + 2\gamma) \varphi_{k, ki} - \epsilon_{ijk} \epsilon_{klm} (\gamma + \varepsilon) u_{m,lj}+ 2\alpha \epsilon_{ijk} u_{k,j} - 4\alpha\varphi + c_i = J \ddot{\varphi}_i\, ,
  \end{align}
\end{subequations}
or, in vector notation
\begin{subequations}\label{eq:disp_vector}
  \begin{align}
    & (\lambda + 2\mu) \grad\div\vb{u} - (\mu + \alpha)\curl\curl\vb{u} + 2\alpha \curl\vb{\varphi} + \vb{f} = \rho \pdv[2]{\vb{u}}{t}, \\
    & (\beta + 2\gamma) \grad\div\vb{\varphi} - (\gamma + \varepsilon) \curl\curl\vb{\varphi} +  2\alpha \curl\vb{u} - 4\alpha\vb{\varphi} + \vb{c} = J \pdv[2]{\vb{\varphi}}{t}\, .
  \end{align}
\end{subequations}

\section{Waves in micropolar solids}
Let's write equation \eqref{eq:disp_vector} in a slightly different way, where we regroup the material constants in a different way and neglect body forces
\begin{subequations}
  \begin{align}\label{eq:disp_waves}
    & c_1^2 \grad\div\vb{u} - c_2^2\curl\curl\vb{u} + K^2 \curl\vb{\varphi} =  \pdv[2]{\vb{u}}{t}, \\
    & c_3^2 \grad\div\vb{\varphi} - c_4^2 \curl\curl\vb{\varphi} +  Q^2 \curl\vb{u} - 2Q^2 \vb{\varphi} = \pdv[2]{\vb{\varphi}}{t}\, .
  \end{align}
\end{subequations}
where,
\begin{equation*}
\begin{split}
c_1^2 = \frac{\lambda +2\mu}{\rho},\quad &c_3^2 =\frac{\beta +2\gamma}{J},\\
c_2^2 = \frac{\mu +\alpha}{\rho},\quad &c_4^2 =\frac{\gamma + \varepsilon}{J},\\
K^2= \frac{2\alpha}{\rho},\quad &Q^2 =\frac{2\alpha}{J} \, ,
\end{split}
\end{equation*}


To identify types of propagating waves that can arise in the micropolar medium we expand our main variables in terms of scalar and vector potentials, as follows
\begin{align*}
\vb{u} &= \grad \phi + \curl\vb{\Gamma}\, ,\\
\vb{\varphi} &= \grad \tau + \curl\vb{E}\, ,
\end{align*}
with
\begin{align*}
&\div\vb{\Gamma} = 0\\
&\div\vb{E} = 0\, .
\end{align*}

If we plug these in \eqref{eq:disp_waves}, we obtain the following, after some manipulations,
\begin{subequations}
  \begin{align}\label{eq:disp_waves}
    c_1^2 \laplacian\phi &= \pdv[2]{\phi}{t}\\
    c_3^2 \laplacian\tau - 2Q^2\tau &= \pdv[2]{\tau}{t}\\
    \begin{bmatrix}
      c_2^2\laplacian &K^2\curl\\
      Q^2\curl &c_4^2\laplacian - 2Q^2
    \end{bmatrix}
    \begin{Bmatrix} \vb{\Gamma}\\ \vb{E}\end{Bmatrix} &=
    \pdv[2]{t} \begin{Bmatrix} \vb{\Gamma}\\ \vb{E}\end{Bmatrix}\, ,
  \end{align}
\end{subequations}
where we can see that the equations for the scalar potentials are uncoupled, while the ones
for the vector potentials are coupled.

If we assume that the potentials are plane waves we can find the dispersion relations. Particularly, for the coupled waves, we have
\begin{align*}
\vb{\Gamma} &= \vb{A}\exp(i\kappa x - i\omega t)\\
\vb{E} &= \vb{B}\exp(i\kappa x - i\omega t)\, ,
\end{align*}
and plugging it into the coupled equation we end up with the following system of equations
\[\begin{bmatrix}
M_{11} &M_{12}\\
M_{21} &M_{22}\end{bmatrix}
\begin{Bmatrix} \vb{A}\\ \vb{B}\end{Bmatrix}
= \vb{0}\, ,\]
with
\begin{align*}
&M_{11} =
(c_{2}^{2} \kappa^{2} - \omega^{2})
\begin{bmatrix}
1 & 0 & 0\\
0 & 1 & 0 \\
0 & 0 & 1
\end{bmatrix}\\
&M_{12} =
\begin{bmatrix}
0 & 0 & 0\\
0 & 0 & i K^{2} \kappa\\
0 & -i K^{2} \kappa & 0
\end{bmatrix}\\
&M_{21} =
\begin{bmatrix}
0 & 0 & 0 \\
0 & 0 & i Q^{2} \kappa \\
0 & -i Q^{2} \kappa & 0 
\end{bmatrix}\\
&M_{22} =
(2 Q^{2} + c_{4}^{2} \kappa^{2} - \omega^{2})\begin{bmatrix}
1 &0 &0\\
0 &1 &0\\
0 &0 &1
\end{bmatrix}\, .
\end{align*}

Since we are not interested in the null solution, we have
\[\det\begin{bmatrix}
M_{11} &M_{12}\\
M_{21} &M_{22}\end{bmatrix}=0\, ,\]
that leads to the dispersion equation 
\begin{equation*}
\omega^2 = Q^{2} + \frac{c_{2}^{2} \kappa^{2}}{2} + \frac{c_{4}^{2} \kappa^{2}}{2} \mp \frac{1}{2} \sqrt{4 K^{2} Q^{2} \kappa^{2} + 4 Q^{4} - 4 Q^{2} c_{2}^{2} \kappa^{2} + 4 Q^{2} c_{4}^{2} \kappa^{2} + c_{2}^{4} \kappa^{4} - 2 c_{2}^{2} c_{4}^{2} \kappa^{4} + c_{4}^{4} \kappa^{4}}\, ,
\end{equation*}
where the minus signs corresponds to the transverse wave and the plus sign corresponds to the rotational wave.

The dispersion relations are then
\begin{align}
\omega_P &= c_1 \kappa\, ,\\
\omega_{RL} &= \sqrt{2Q^2 + c_3^2 \kappa^2}\, ,\\
\omega_S &= \sqrt{Q^{2} + \frac{(c_2^2 + c_4^2)}{2}\kappa^2 - \frac{1}{2} \sqrt{4Q^4 +
   4Q^2[(c_4^2 - c_2^2) + K^2]\kappa^2 + (c_4^2 - c_2^2)^2 \kappa^4}}\, ,\\
\omega_{RT} &= \sqrt{Q^{2} + \frac{(c_2^2 + c_4^2)}{2}\kappa^2 + \frac{1}{2} \sqrt{4Q^4 +
   4Q^2[(c_4^2 - c_2^2) + K^2]\kappa^2 + (c_4^2 - c_2^2)^2 \kappa^4}}\, ,
\end{align}
where we can see that the only wave that is not dispersive is the P-wave, since the relationship between wavenumber and frequency is linear.



\bibliographystyle{unsrtnat}
\bibliography{bloch-cosserat}

\end{document}
